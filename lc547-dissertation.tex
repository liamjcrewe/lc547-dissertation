\documentclass[11pt,openright,a4paper]{report}

\include{DissertationDefs}    %% These are the includes required for the doc 

\usepackage{hyperref}
\usepackage[parfill]{parskip}
\usepackage{csquotes}
\usepackage{graphicx}
\usepackage[toc, page]{appendix}
\usepackage[
  backend=bibtex,
  style=authoryear,
  citestyle=authoryear,
  dateabbrev=false
]{biblatex}

% Biblatex long urls fix
\makeatletter
\def\blx@maxline{77}
\makeatother

% Fix to apply hyperref to whole biblatex citation instead of just year.
\DeclareCiteCommand{\parencite}[\mkbibparens]
  {\usebibmacro{prenote}}
  {\usebibmacro{citeindex}%
    \printtext[bibhyperref]{\usebibmacro{cite}}}
  {\multicitedelim}
  {\usebibmacro{postnote}}

\DeclareCiteCommand*{\parencite}[\mkbibparens]
  {\usebibmacro{prenote}}
  {\usebibmacro{citeindex}%
    \printtext[bibhyperref]{\usebibmacro{citeyear}}}
  {\multicitedelim}
  {\usebibmacro{postnote}}
% End biblatex fix

\title{Development of an extensible personal informatics system to aid in self management of mental health and wellbeing, making use of both user provided data and user specific API data}
\author{Liam Crewe}
\date{Bachelor of Science in Computer Science with Honours\\University of Bath\\May 2017}

\addbibresource{bibtex.bib} 

\begin{document}

% Set this to the language you want to use in your code listings (if any)
\lstset{language=Java,breaklines,breakatwhitespace,basicstyle=\small}

\setcounter{page}{0}
\pagenumbering{roman}

\maketitle
\newpage


% Set this to the number of years consultation prohibition, or 0 if no limit
\consultation{0}
\newpage


\declaration{Development of an extensible personal informatics system to aid in self management of mental health and wellbeing, making use of both user provided data and user specific API data}{Liam Crewe}
\newpage


\abstract
Your abstract should appear here.  An abstract is a short
paragraph describing the aims of the project, what was
achieved and what contributions it has made.
\newpage


\tableofcontents
\newpage
\listoffigures
\newpage
\listoftables
\newpage


\chapter*{Acknowledgements}
Add any acknowledgements here.
\newpage


\setcounter{page}{1}
\pagenumbering{arabic}

\chapter{Introduction}
\section{Background}
\subsection{Mental Health in the UK}
One in four adults (around 26\%) have been diagnosed with at least one mental health condition \parencite{hse2014} (where adult is any person 16 or over). A further 18\% reported to have experienced a mental health condition but did not seek a diagnosis \parencite{hse2014}. This equates to around 44\% of adults in the UK. The population of the UK is around 65 million \parencite{onspopulation}, of which around 18.8\% are under the age of 16 \parencite{onspopulation}. This equates to a population over the age of 16 of around 81.2\%, or around 52.78 million people. Given that 44\% of these will experience a mental health condition, this equates to over 23 million adults in the UK alone suffering from mental health conditions. This, in addition to any children and young people with mental health conditions truly shows the prevalence of mental health conditions in the UK.

This figure is further supported by the fact that mental health is the result of over 70 million sick days every year \parencite{cmoreport2013}, and is the largest burden of disease with around 28\% of the total burden, compared to 16\% for each of cancer and heart disease \parencite{burdendisorders}. Mental health conditions are estimated to cost the economy between £70 and £100 billion every year \parencite{cmoreport2013}.

\subsection{Mental Health Funding and Capacity}
Despite these figures, the funding for mental health services only sums to around 13\% of total NHS spending \parencite{cepnhsfunding}. In fact, around 75\% of people with mental health conditions receive no treatment at all \parencite{cmoreport2013}. On top of this, although the NHS (National Health Service) is due to receive increased funding in 2016 and the years that follow \parencite{kfnhsbudget}, funding for mental health services is due to rise by just 0.3\% \parencite{mhfunding}. 53 out of 59 services in the UK responded to the freedom of information act made as part of the cited BBC article by \citeauthor{mhfunding}; 23 of which said that their funding would in fact decrease in 2016.

\section{Problem Description}
\subsection{Self Management in Mental Health}
This project is not aiming to find a way to increase the supply of treatment from mental health services, but rather to reduce the demand for it. Much research has been done into the effectiveness of self management for mental health conditions. For example, a study in 2011 tested a particular self management scheme and found that it \enquote{reduces psychiatric symptoms, enhances participants’ hopefulness, and improves their QOL over time} \parencite{wrapstudy}. This shift to self management could help ease the pressure on mental health services, by relieving some of the demand and allowing users to manage their conditions before they worsen. It has also proven particularly effective for relapse prevention in people who have previously undergone treatment for mental health conditions, in that it has been described to \enquote{play a critical role in people's recovery from mental illness} \parencite{selfmanagementrelapse}.

\subsection{Personal Informatics in Mental Health}
This relates to a field known as personal informatics. \enquote{Personal informatics is a class of tools that help people collect personally relevant information for the purpose of self-reflection and self-monitoring} \parencite{personalinformatics}. Personal informatics in mental health is an interesting topic. In 2011 a project aimed to create a personal informatics system that combined sensor data with user inputted data, with the aim of allowing self management of bipolar disorder \parencite{pimentalhealth}. This study found that although personal informatics systems had clear relevance to mental health treatment, little research had been done in the area. It also concluded that the if a system could fuse the collection of personal data into treatment tools in some way, this may have relevance to other mental health disorders, as well as more mild issues such as stress management and the building of balanced, productive lifestyles.

Various applications already exist to allow self management of mental health conditions, such as Optimism \parencite{optimism}. This works by collecting user inputted data and building visualisations the user can look at and draw conclusions from.

While this method of user inputted data works, it is flawed for several reasons. Firstly, users may not be aware of what is a dangerous pattern \parencite{pimentalhealth}, and secondly it requires users to fill in (sometimes lots of) fields fairly frequently. This could lead users to not use the system as often as they could and should. This could be avoided; many modern applications already record data about users, either via user input or automatically. There are many examples of these: mental health apps such as Optimism \parencite{optimism}, fitness apps such as Strava \parencite{strava}, social media apps such as Twitter \parencite{twitter} and many others. This results in a huge amount of data being recorded and made available from applications the user may already use, without requiring the user to submit the data themselves.

This data is very important to this project as mental health is affected by such a wide variety of factors. The examples above for example are discussed below:

\subsubsection{Exercise}
It is common to recommend exercise as a way to help with various mental health conditions. Physical activity is known to alleviate symptoms of mild to moderate depression and has been associated with improvement of self-concept and confidence, as well as reduced symptoms of anxiety \parencite{exercisementalhealth}. Various apps such as Strava \parencite{strava} track how much and how often a user exercises, and could allow the user to correlate exercise frequency and duration with mental health issues.

\subsubsection{Social Media}
It has been shown for example that Facebook can cause depression in some teens when used excessively \parencite{fbdepressionteens}. A research team at Microsoft was even able to build a classifier that used Twitter data to predict onset depressive episodes in patients with major depression, with an accuracy of around 70\% \parencite{de2013predicting}.

\subsection{This Project}
\subsubsection{Overview}
This project aims to take advantage of this data by producing an application called SelfReflect. SelfReflect will only require the user to record a small amount of data about themselves. This will consist of a simple wellbeing questionnaire, which will calculate a wellbeing score. This will then be recorded, along with date and time of submission. The exact nature of this questionnaire will be discussed later, but the key part is that the questionnaire produces a reliable mental health wellbeing score, and is simple, fast and easy for the user to fill out, with the goal that this will allow the user to record their wellbeing more often.

SelfReflect will consist of three distinct parts: a mobile app to allow recording of wellbeing (and automatically date and time of submission), a server side application programming interface (API) to allow storing and retrieval of user data and a web app. The web app will allow the user to add credentials (via the SelfReflect API) to various apps such as Twitter and Strava. This will allow SelfReflect to pull data from these apps via their individual APIs. The web app will then allow the user to interrogate this data and produce visualisations across the multiple data sources.

The goal of this is not to produce a system with an extensive variety of sources and visualisations for the user. Entire projects could be completed on producing lots of different visualisations from just one of these sources with regard to mental health. Instead, SelfReflect will be developed to be extensible; to be built on in the future. Support for two sources (Twitter and Strava) will be developed, with one visualisation for each, as \enquote{proofs of concept}, but this is by no means the final goal of the software system. The API will be built in such a way that a user (or developer) can fetch their data from each source via the SelfReflect API (given that they have provided valid credentials), providing an \enquote{API of APIs}. Also, the API will return \emph{all} data it can access for the user from that source, rather than having one endpoint per visualisation, allowing multiple visualisations to be built from each source in the future.

\section{Aims}
\begin{itemize}
\item Design a system to aid self management of mental health by combining user submitted data with existing applications data., and producing visualisations from this combined data.
\item Design this system to be extensible, so that it can be built on in the future via the addition of more options for sources, more visualisations for new and/or existing sources.
\item Test this system in both a functional and user experience context, including the usability of the various aspects of the system.
\item Investigate the feasibility of this tool as an aid for self management of mental health.
\end{itemize}

\section{Objectives}
\begin{itemize}

\item Design and implement an API that allows:
\begin{itemize}
  \item User creation, log in and management.
  \item Storage of wellbeing score, as well as date and time of submission.
  \item Fetching of user's wellbeing scores over time.
  \item Connection of existing applications (Twitter and Strava) when given valid applications credentials for a user, and storage of these credentials.
  \item Fetching of user's existing application data, to create an \enquote{API of APIs}
\end{itemize} 
  
\item Design and implement a mobile app that allows:
\begin{itemize}
  \item User creation and log in.
  \item Recording of wellbeing score, as well as date and time of submission.
\end{itemize}

\item Design and implement a web application that, via the API disscussed above, allows:
\begin{itemize}
  \item User creation, log in and management.
  \item Recording of wellbeing score, as well as date and time of submission.
  \item Connection of existing applications when given valid applications credentials for a user.
  \item Fetching and combining of user's data from both SelfReflect and connected existing applications, to produce visualisations from this data (one for each of the \enquote{proof of concept} sources.
  \item 
\end{itemize} 

\item Distribute this application to a series of volunteer testers, and get feedback on both the functionality of the system (i.e. if it works) and the usability of the various areas of the system.
\item Analyse the application and the results of the testing to:
\begin{itemize}
  \item Identify possible future changes that could be made and functionality/features that could be added in order to improve the system.
  \item Identify the effectiveness of the sources and visualisations system, and improvements that can be made to it.
  \item Discuss the extensibility of this application for future development.
  \item Discuss the feasibility of this application as a tool for self management of mental health.
\end{itemize}

\end{itemize}

%%%%%%%%%% @TODO:
%% May need to update this later. Depends what goes in each section.
\section{Dissertation Structure}
\textbf{Chapter 1: Introduction}

This chapter provides some high-level background research and motivation for the project, and gives an overview of the project, including its aims and objectives.

\textbf{Chapter 2: Literature Survey}

This chapter delves deeper into the research relevant to this project. It considers relevant literature in terms of motivations and system design, researches methods of recording wellbeing, and compares and contrasts existing applications.

\textbf{Chapter 3: Requirements Specification}

This chapter details the functional and non functional requirements for the system, as derived from the project aims and objectives, and the literature survey. It also discusses the gathering, development and terminology of these requirements.

\textbf{Chapter 4: Design}

This chapter details the design of the three parts of the system (API, mobile app and web app), in order to fulfil the requirements as defined in chapter 3.

\textbf{Chapter 5: Implementation}

This chapter details the implementation of the design as defined in chapter 4.

\textbf{Chapter 6: Testing}

This chapter first discussed automated testing, then continues to detail the methods used for testing functionality, usability and feasibility of the system, using volunteer testers.

\textbf{Chapter 7: Results and Discussion}

This chapter presents the results of the testing by the volunteer testers. It then discusses these results, improvements that could be made to the system based on these results, and the feasibility of the system as a tool to aid in self management of mental health and wellbeing.

\textbf{Chapter 8: Conclusion and Future Work}

This chapter summarises the project and discusses its successes and failures. It also details areas for improvement and for future work.

%%%%%%%%%%

\chapter{Literature Survey}
\section{Introduction} \label{introduction}
This literature survey will identify the various fields this project relates to. It will then present and critique existing work related to these fields, both in the form of academic research and existing technologies. By doing so, it will discuss what exactly this project hopes to contribute to these fields, how it plans to do so, and how this contribution would be useful.

This project encompasses two key fields: personal informatics and mental health. These are two active areas of research, as will be shown and discussed in later sections. This project hopes to combine these two fields in ways that will contribute to them, both in an academic sense, and with the development of a software system that encompasses both of these fields, to aid people in the effective self management of their mental health and wellbeing.

The existing research in these fields will be extremely useful in the design and development of the system and its requirements. These will be developed by drawing on the conclusions of existing work, and by making improvements on existing technologies.

\subsection{Self-management in mental health}
Self-management techniques and schemes have been shown to be effective in both treatment \parencite{wrapstudy} and recovery \parencite{selfmanagementrelapse}. Many schemes exist today and are growing in popularity, with these schemes being said to offer an \enquote{increasingly popular alternative to therapist-administered psychological therapies, offering the potential of increased access to cost-effective treatment} \parencite{selfhelpanxiety}.

Much of the principle of self-management revolves around \enquote{empowerment} of the user \parencite{whoselfmanagement}. Empowerment involves the user in decisions about his or her mental health condition treatment, providing them with a level of control and influence \parencite{whoempowerment}. Although self-management and active engagement has been commonplace for physical long term health conditions for a long time, it has been much less widely used in the context of mental health \parencite{whoselfmanagement}. It is only fairly recently that this has become common in mental health treatment, with claims that it can be as effective as medical treatment \parencite{mhfselfmanagement}.

For a long time, mental health patients were thought of as passive recipients of care, but attitudes have changed over recent years \parencite{cpselfmanagement}. It was discovered that empowering users significantly improved treatment and recovery, giving users the \enquote{knowledge, skills, and self-confidence} to \enquote{take care of their own health and manage to live life competently} \parencite{vahdatpatientinvolvement}.

Many self-management schemes are based around teaching users specific skills, and providing them with the tools and techniques required to manage their mental health by themselves \parencite{selfmanagementuk}. The amount of clinical intervention required varies depending on the scheme, ranging from pure self help techniques (no clinical involvement) to guided self help (some clinical involvement), be that through face-to-face sessions, telephone or email \parencite{selfhelpanxiety}.

There are lots of self-management techniques, many of which are common recommendations to those suffering from mental health conditions or poor mental health. Some of these common techniques will be discussed below, with regard to how they apply to this project, and how this project can aid execution of each technique.

\subsubsection{Cognitive Behavioural Therapy (CBT) techniques}
Cognitive Behavioural Therapy, or CBT, is a form of talking therapy that is extremely popular in modern mental health treatment. It is \enquote{one of the most extensively researched forms of psychotherapy}, which is \enquote{due in part to the ongoing adaptation of CBT for an increasingly
wider range of disorders and problems} \parencite{butler2006empirical}.

Although CBT itself is not a self-management technique, it involves users learning mental health management tools, known as CBT techniques, that can be used in daily life after treatment ends \parencite{babcpcbt}. CBT works by changing how a person interprets or reacts to things that happen to or around them, and teaching them to identify harmful, negative or unbalanced reactions \parencite{nhscbt}. This is in itself a self-management technique, and has been said to be one of the main reasons CBT has such impressively low relapse rates. A study by \citeauthor{butler2006empirical} (\citeyear{butler2006empirical}) revealed that for patients treated for depression, on average \enquote{only 29.5\% of CBT patients relapsed versus 60\% of patients treated with antidepressants}.

CBT is a combination of two types of therapy: cognitive and behavioural \parencite{patientcbt}. Cognitive techniques relate to negative thoughts, and to what a person thinks, or how a person reacts emotionally to a situation or event \parencite{medscapecbt}. It is be unlikely that the system developed in the project will be able to directly measure the effectiveness of cognitive techniques.

Behavioural techniques however often involves substituting alternate, less harmful behaviours for harmful ones \parencite{patientcbt}. If this behaviour is one measured by a system, such as exercise or social activity, its effectiveness could be measured by the system and reported to the user. This could allow the user to identify which CBT techniques are more useful to them, as well as the effect on their mental health of not employing these techniques. They could also identify when a technique or techniques become less effective, and either address this themselves or recognise this as a sign of relapse, and consider contacting a mental health clinician before their condition worsens.

\subsubsection{Trigger identification}
Triggers refer to events that can potentially result in relapses or negative impacts on mental health \parencite{samhsatriggers}. These events can \enquote{trigger} these issues. Users identify triggers and plan their responses in advance so that they can react appropriately to try to avoid negative consequences \parencite{samhsatriggers}.

The aim of this system in regard to triggers is that, should a user experience a negative change in mood, they can view the data around that time and identify potential triggers that resulted in that negative change. They can also, in a similar way to CBT techniques, measure the effectiveness of their planned actions to triggers. If the user knows a trigger occurred, and a planned action was carried out, they can investigate how effective this reaction was in avoiding the negative consequences of the trigger.

\subsubsection{Wellness Recovery Action Plan (WRAP)}
Wellness Recovery Action Plan, or WRAP, is another popular and widely used system for self-management \parencite{mhrwrap}. There is some crossover with the above methods. A user creates their own WRAP, which includes (definitions taken from \parencite{mhrwrap}):
\begin{itemize}
\item A \enquote{Wellness Toolbox}. That is, resources such as contacting friends, exercising and maintaining a healthy diet.
\item A daily maintenance plan; tasks to do every day to maintain wellness.
\item Triggers; as discussed above.
\item Early warning signs of worsening symptoms.
\item A plan for when things are breaking down, or when things are getting much worse.
\item A crisis plan; signs that inform others your care needs to be taken over, and what they should do.
\item A post crisis plan; what to do to get yourself well again.
\end{itemize}

This clearly has some crossover with the above methods. Again, the system developed in this project would in theory be able to help with this. An early warning sign, or even later sign could be identified by the wellbeing data recorded in the system. The effectiveness of the daily maintenance plan and post crisis plan could also be measured, and potentially updated depending on the results of the data recorded. This also applies to the Wellness Toolbox.

\subsubsection{Self-management schemes similarities}
All three of these schemes have some common features. They all require:
\begin{itemize}
\item Identification/recognition of events that can cause negative consequences to mental health.
\item Recognition of when these events happen and the effect of them.
\item Planned actions to minimise/counteract the negative consequences of these events.
\end{itemize}

To do this, and to measure the effectiveness of these actions, some measurement of mental wellbeing needs to be made regularly. In theory, this would be best if made both before and after the trigger, as well as after the planned action (or a while after the trigger if no planned action is taken).

This is particularly important for the purposes of this system. Only if this measurement is effective, efficient and recorded regularly will the system be able to aid in these self-management techniques. This is another area that requires research and a review of existing work, in order to decide the best way to allow the user to record their wellbeing easily and effectively.

\subsection{Recording of mental health and wellbeing} \label{recordingmentalhealth}
\subsubsection{Method of recording} \label{methodofrecording}
There are two key ways to record a person's wellbeing: self-rating and observer rating. For the purposes of this project, self-rating will be used, as the application is designed for self-management. Regardless of the fact that this is necessary for this project, self-rating methods may well be more effective anyway. A study in 2011, published in the journal of mental health, investigated the effectiveness of various measures from the point of view of service users \parencite{crawford2011selecting}.

This study highlighted the importance service users place on measures rated by the patient. The participants in the study even \enquote{expressed surprise, and in some instances disbelief, that outcome measures based entirely on the judgements of researchers or clinicians could be used to judge a person’s response to treatment}. This study will be particular useful when discussing how to record mental health and wellbeing.

\subsubsection{Ecological momentary assessment (EMA)}
Ecological momentary assessment (EMA) refers to a method of data collection in which a person can \enquote{report on symptoms, affect and behaviour close in time to experience} \parencite{moskowitz2006ecological}. This strongly relates to this project, as a key aim is to allow users to record their wellbeing regularly and easily, allowing EMA. EMA has \enquote{a number of advantages over more traditional methods for the assessment of patients}, and has been shown to \enquote{permit more sensitive assessments} and \enquote{enable more wide-ranging and detailed measurements of mood and behaviour} \parencite{moskowitz2006ecological}.

\subsubsection{Questionnaire length and response burden} \label{questionnairelength}
Another important consideration is the length of the method used to record wellbeing. This relates to a concept known as \enquote{response burden}. That is, the effort required by a user to answer a questionnaire \parencite{rolstad2011response}. The initial proposal for this project was for the questionnaire length to be just one field, rating a mood or wellbeing on a simple scale. However, research shows that users accused questionnaires that are too short of not being able to \enquote{properly assess the complex
outcomes they are designed to measure} \parencite{crawford2011selecting}.

Questionnaires that are too long however have been shown to result in a lower response rate, and faster, shorter, more uniform responses to later questions \parencite{galesic2009effects}. This is possibly due to the questionnaire having too large a response burden for the user. The user may become bored or disillusioned with the measure and think less about responses.

A high response burden poses another problem for this project, in that this system aims to allow users to record their wellbeing quickly and easily. A higher response burden could lead to the user delaying the recording of their wellbeing, which would then not count as EMA. To overcome this, the system must implement some trade off between response burden, and obtaining meaningful measures of mental health and wellbeing.

\subsubsection{Outcome measures}
In both mental and physical health, measurements of a patient's progress and the efficacy of their treatment are often referred to as \enquote{outcome measures} \parencite{pediaoutcomemeasures}. These measures are said to allow health services to improve quality of service, but can also be used in the context of tracking a patient's mental health and wellbeing. These measures come in many forms, and fall into various possible categories: general outcome measures, patient rated outcome measures (PROMS), outcome measures for use with children, outcome measures for emotional difficulties and various others \parencite{mhpoutcomemeasures}. As discussed in section \ref{methodofrecording} above, we are only interested in self-rating methods, and hence will only investigate patient reported outcome measures.

\subsection{Which outcome measure(s)?} \label{whichoutcomemeasures}
There are many condition specific outcome measures, for example the \enquote{Patient Health Questionnaire} (PHQ-9) for depression \parencite{kroenke2001phq}, and the \enquote{Generalized Anxiety Disorder} (GAD7) questionnaire for generalised anxiety disorder \parencite{spitzer2006brief}.

These condition specific questionnaires may well be more effective than generic wellbeing questionnaires. This project however will focus on the more generic examples. This is due to the fact that the key goal of this project is to investigate the combinations of user provided wellbeing data with user specific API data. The scope of the project does not allow for investigation and development of a new or custom outcome measure. Neither does it allow for the inclusion of multiple condition specific outcome measures. A key goal of the application is that it is generic, and at least somewhat applicable to the majority of the general population, in regards to keeping track of their mental health and wellbeing.

For these reasons the project will implement an existing, well established patient reported outcome measure. Some examples of existing, popular outcome measures will be discussed, as will their appropriateness for use in this project. Candidates to be discussed (one of which will be included in the system) are: the \enquote{European Quality of Life instrument} (EQ-5D), the \enquote{Recovery Star}, the \enquote{Warwick-Edinburgh Mental Wellbeing Scale} (WEMWBS), and its short version, the \enquote{Short WEMWBS} (SWEMWBS).

\newpage
\subsubsection{European Quality of Life instrument (EQ-5D)}
The \enquote{European Quality of Life instrument} (EQ-5D) is a \enquote{standardized instrument for use as a measure of health outcome} \parencite{eq5dabout}. The questionnaire consists of two pages. The user answers 5 questions on a scale from \enquote{no problem} to \enquote{unable to do} in areas such as mobility and self care, then scores their mood on a scale of 1 to 100 \parencite{eq5duse}. An example EQ-5D can be seen in Figure \ref{fig:eq5d} (images from \parencite{eq5duse}):

\begin{figure}[ht]
\caption{The European Quality of Life instrument}
\includegraphics[width=.5\textwidth]{i/eq5d1.jpg}\hfill
\includegraphics[width=.5\textwidth]{i/eq5d2.jpg}
\label{fig:eq5d}
\end{figure}

The EQ-5D has been shown to be useful for common mental health conditions such as mild to moderate depression \parencite{brazier2010eq}, which may well be appropriate for this system. However, it is possible it suffers from an issue discussed in the section \ref{questionnairelength} above; it is too short. In fact, in the study by \citeauthor{crawford2011selecting}, it was this questionnaire that participants raised concerns about with regard to it being too short to properly assess complex outcomes \parencite{crawford2011selecting}.

\newpage
\subsubsection{Mental Health Recovery Star (MHRS)}
The \enquote{Mental Health Recovery Star} (MHRS) is part of a set of outcome measure tools called \enquote{Outcomes Stars} \parencite{outcomesstars}. It measures ten areas of a user's life, rating them on a scale of 1 to 10, where 1 is \enquote{stuck} and 10 is \enquote{self-reliance}. This then produces a graphic in the shape of a ten-pointed star, an example of which can be seen in Figure \ref{fig:mhrs} (image from \parencite{outcomesstars}):

\begin{figure}[ht]
\centering
\caption{The Mental Health Recovery Star}
\includegraphics[width=\textwidth]{i/recoverystar.jpg}]
\label{fig:mhrs}
\end{figure}

This measure has been \enquote{received enthusiastically by both
mental health service providers and service users} \parencite{dickens2012recovery}. A study by \citeauthor{killaspy2012psychometric} aimed to analyse the psychometric properties of the MHRS \parencite{killaspy2012psychometric}. It found some positive results, revealing that 85\% of participants found the MHRS useful in areas such as planning support and tracking their recovery. It also found that 70\% of participants said the MHRS was easy to use. However, this study only considered MHRS recorded either by staff or collaboratively with staff and user.

Staff or collaborative recording is common with the MHRS; many staff are given training in order to guide users in using the measure. In fact, training is now a requirement to gain a license to use the star \parencite{starlicense}. Clearly this poses an issue for this project. Assuming a license could be obtained without training (which may be unlikely), it is possible that this training issue could be overcome with computerised tutorials and careful user experience design. However, this would still require the user to learn how to use the MHRS, and would be more complicated than a more simple, intuitive outcome measure.

Another issue with MHRS is the time to complete. In the study by \citeauthor{killaspy2012psychometric} it was found that it, on average, took 30 to 60 minutes to complete \parencite{killaspy2012psychometric}. While this was posed as a positive in this study, it is likely a negative for the purposes of this project. Response burden needs to be considered with regard to the desired frequency of response. For this project, the aim is to have a high frequency of response, meaning users will be able to regularly record their wellbeing. This could be multiple times per week, once per day or even multiple times per day. 

When considering 30 to 60 minutes with regard to this high desired response frequency, this is quite a high response burden. As discussed in the questionnaire length and response burden section, this is a key problem for this project. Also, it is likely that this time will be even longer than 30 to 60 minutes, given that the user would be completing the MHRS on their own without the guidance of a trained member of staff.

\subsubsection{Warwick-Edinburgh Mental Wellbeing Scale (WEMWBS)}
The \enquote{Warwick-Edinburgh Mental Wellbeing Scale} (WEMWBS) is a scale designed to \enquote{enable the monitoring of mental wellbeing in the general population} \parencite{wemwbs}. It was developed by a panel of experts \parencite{wemwbsdevelopment} and has been tested fairly extensively, finding that it \enquote{showed good content validity} and is a \enquote{short and psychometrically robust scale} \parencite{tennant2007warwick}. It consists of 14 positively worded items with 5 possible responses to each, ranging from \enquote{none of the time} to \enquote{all of the time}.
\newpage
These questions and responses can be seen in Figure \ref{fig:wemwbs} (table from \parencite{wemwbsquestions}):
\begin{figure}[ht]
\centering
\caption{The Warwick-Edinburgh Mental Wellbeing Scale}
\includegraphics[width=.8\textwidth]{i/wemwbs.png}
\label{fig:wemwbs}
\end{figure}

The fact that WEMWBS was developed for the general population is particular useful to this project, as a general purpose, simple scale is exactly what this project requires. Another strength of this scale is that it results in a single score, which is simply a sum of the responses. Each response is scored 1-5, giving a score of 14-70 for the full questionnaire \parencite{wemwbsscoring}. This is beneficial to this project as a single score can be easily recorded and mapped against other data such as date/time, or user specific API data such as social media activity.

The study by \citeauthor{crawford2011selecting} found the WEMWBS to be largely successful, in that it achieved one of the highest ratings among participants \parencite{crawford2011selecting}. This study identified another key problem with various outcome measures. This was the use of negatively phrased questions, as users said it can be \enquote{upsetting to be asked long lists of questions about difficulties associated with mental ill
health}. The WEMWBS was commended by participants for phrasing items positively, where poor mental health would be recognised by a lack of these positive terms, rather than the presence of negative terms.

A problem with WEMWBS however is the length of the questionnaire, in that it consists of 14 items. This shouldn't be too time consuming for the user as the questions are not open ended, and are relatively simple. However, again regarding the aim of a high response frequency, a shorter questionnaire may be more effective for the purposes of this project. Thankfully, a shorter version of the WEMWBS exists; the \enquote{Short WEMWBS} (SWEMWBS).

\subsubsection{Short WEMWBS (SWEMWBS)}
The \enquote{Short WEMWBS} (SWEMWBS) is a shorter version of the WEMWBS, consisting of a 7 item scale rather than a 14 item scale \parencite{swemwbs}. The questions on the short version are a subset of those on the full version, so retain the advantages of being positively worded. In order to compare the two scales, the score obtained from the shorter version is mapped to an equivalent longer version score via a conversion table. This conversion table can be seen in appendix \ref{swemwbsconversiontable}.
\newpage
The questions in the SWEMWBS can be seen in Figure \ref{fig:swemwbs} (table from \parencite{swemwbsquestions}):
\begin{figure}[ht]
\centering
\caption{The Short Warwick-Edinburgh Mental Wellbeing Scale}
\includegraphics[width=.8\textwidth]{i/swemwbs.png}
\label{fig:swemwbs}
\end{figure}

The scale was developed in a research project by \citeauthor{stewart2009internal} (\citeyear{stewart2009internal}). This project tested the shorter scale and drew various conclusions on its validity and effectiveness. While the longer version of course provides more detail, the r-value was found to be 0.954 in the development and testing of the scale, meaning the scores produced are very similar. Given the high correlation of scores, the lower response burden and robust measurement properties of SWEMWBS, this testing found that this made \enquote{SWEMWBS preferable to WEMWBS at present for monitoring mental wellbeing in populations} \parencite{stewart2009internal}.

The disadvantages of this scale are that, as discussed, it does provide a more narrow view of a person's wellbeing. \citeauthor{stewart2009internal} (\citeyear{stewart2009internal}) found that where \enquote{face validity is an issue there remain arguments for continuing to collect data on the full 14 item WEMWBS}. However, for the purposes of this project a simpler scale with lower face validity is preferable.

The SWEMWBS (as well as the WEMWBS) is copyrighted by the University of Warwick and NHS Health Scotland. However, it is free to use as long as users register by completing a registration form \parencite{wemwbsreg}.

\subsubsection{This project}
The SWEMWBS will be used for this project as it seems to provide a good trade off between response burden and obtaining meaningful measures of mental health and wellbeing. It is publicly available (after registration) and provides a single wellbeing score per response, which will aid massively when building visualisations. This will make mapping of wellbeing to other factors simple, which should result in simple, effective and easy to understand visualisations for the user.

\section{Personal informatics} \label{personalinformatics}
\subsection{Introduction}


\chapter{Requirements Specification}

\ldots


\chapter{Design}

\ldots


\chapter{Implementation}

\ldots


\chapter{Testing}

\ldots


\chapter{Results and Discussion}

\ldots


\chapter{Conclusions and Future Work}

\ldots

\printbibliography

\begin{appendices}
\section{SWEMWBS conversion table} \label{SWEMWBS conversion table}
\begin{figure}[ht]
  \centering
  \includegraphics[width =.7\textwidth]{i/swemwbsconversiontable1.png}
  \label{swemwbsconversiontable}
\end{figure}

\begin{figure}[ht]
  \centering
  \includegraphics[width =.7\textwidth]{i/swemwbsconversiontable2.png}
\end{figure}
\end{appendices}

\end{document}






%% EXAMPLES

%\chapter{Example Introduction}
%%% Uncomment this to include a separate tex file wih the introduction contents
%%\include{introduction.tex}
%
%This is the introductory chapter.
%
%\section{Example Section}
%Like all chapters, it will have a number of sections
%
%\subsection{Example Subsection}
%\ldots and sub-sections
%
%\subsubsection{Example sub-subsection}
%\ldots and sub-subsections.
%
%\begin{table}[htb]
%\begin{center}
%\caption{An example table}
%\label{Example-Table}
%\begin{tabular}{|l|l|}
%\hline
%Items & Values \\
%\hline
%\hline
%Item 1 & Value 1 \\
%Item 2 & Value 2 \\
%\hline
%\end{tabular}
%\end{center}
%\end{table}
%
%\section[short section title]{Another section}
%Another section, just for good measure.
%You can referene a table, figure or equation using \verb|\ref|, just
%like this reference to table \ref{Example-Table}.
%
%\section{Example lists}
%
%\subsection{Enumerated}
%
%\begin{enumerate}
%\item Example enumerated list
%  \begin{itemize}
%  \item a nested enumerated list item
%  \end{itemize}
%\item Second item in the list
%\end{enumerate}
%
%\subsection{Itemized}
%
%\begin{itemize}
%\item Example itemized list
%  \begin{itemize}
%  \item a nested itemized list item
%  \end{itemize}
%\item Second item in the list
%\end{itemize}
%
%\subsection{Description}
%
%\begin{description}
%\item[Item 1] Example description list
%\item[Item 2] Second item in the list
%\end{description}
%
%
%\chapter{Literature Survey}
%%% Uncomment this to include a separate tex file wih the introduction contents
%%\include{litsurvey.tex}
%This is the chapter for your Literature Survey.
%
%You will wish to cite authors like \parencite{latex} or \parencite{btxdoc}. 
%%Alternate commands are used to cite \citeasnoun{latex} as a noun, or cite
%%\possessivecite{latex} work possessively, or add text to the citation, 
%%\citeaffixed{latex}{e.g.}.
%
%If these citations do not compile correctly, ensure you have the Harvard
%package installed.  You can pick up the Harvard package in the zip file
%of the dissertation template files you downloaded.
%
%
%%%
%%% NOTE: Replace the following with chapters that are appropriate for your
%%%       style of project.  It is unlikely these will fit your project perfectly.
%%%
%
%\chapter{Requirements}
%If you are doing a primarily software development project, this is the
%chapter in which you review the requirements decisions and
%critique the requirements process.
%
%
%\chapter{Design}
%This is the chapter in which you review your design decisions at various
%levels and critique the design process.
%
%
%\chapter{Implementation}
%This is the chapter in which you review the implementation and testing
%decisions and issues, and critique these processes.
%
%Code can be output inline using \verb@\lstinline|some code|@.  For example,
%this code is inline: \lstinline|public static int example = 0;|  (I have
%used the character \verb@|@ as a delimiter, but any non-reserved character
%not in the code text can be used.)
%
%Code snippets can be output using the \verb|\begin{lstlisting} ... \end{lstlisting}|
%environment with the code given in the environment.  For
%example, consider listing \ref{Example-Code}, below.
%
%\begin{lstlisting}[breaklines,breakatwhitespace,caption={Example code},label=Example-Code]
%public static void main() {
%
%  System.out.println("Hello World");
%
%}
%\end{lstlisting}
%
%Code listings are produced using the package ``Listings''.  This has many
%useful options, so have a look at the package documentation for further
%ideas.
%
%\chapter{Testing}
%
%\chapter{Results and Discussion}
%This is the chapter in which you review the outcomes, and
%critique the outcomes process.  You may include user evaluation here
%too.
%
%
%%%
%%% Now we are back to the standard project contents that you should include
%%%
%
%\chapter{Conclusion and Future Work}
%%% Uncomment this to include a separate tex file wih the conclusion contents
%%\include{conclusion.tex}
%
%This is the chapter in which you review the major achievements in the
%light of your original objectives, critique the process, critique your
%own learning and identify possible future work.
%
%\printbibliography
%
%\begin{appendices}
%
%%%
%%% Use the appendix for major chunks of detailed work, such as these. Tailor
%%% these to your own requirements
%%%
%
%\chapter{Design Diagrams}
%
%\chapter{User Documentation}
%
%\chapter{Raw results output}
%
%\chapter{Code}
%
%%%
%%% NOTE that for this to typeset correctly, ensure you use the pdflatex
%%%      command in preference to the latex command.  If you do not have
%%%      the pdflatex command, you will need to remove the landscape and
%%%      multicols tags and just make do with single column listing output
%%%
%
%\begin{landscape}
%\begin{multicols}{2}
%\section{File: yourCodeFile.java}
%\lstinputlisting[basicstyle=\scriptsize]{yourCodeFile.java}
%\end{multicols}
%\end{landscape}
%
%\end{appendices}
%
%\end{document}
